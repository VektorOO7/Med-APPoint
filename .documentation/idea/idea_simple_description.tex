\documentclass[a4paper,12pt]{article}

\usepackage[margin=1in]{geometry}
\usepackage{xltabular}
\usepackage[utf8]{inputenc}
\usepackage[T2A]{fontenc}
\usepackage[bulgarian]{babel}
\usepackage[table,xcdraw]{xcolor}
\usepackage{textcomp}

\geometry{margin=2cm}

\definecolor{Color1}{RGB}{250, 191, 143}
\definecolor{Color2}{RGB}{253, 233, 217}

\begin{document}

\begin{xltabular}{\textwidth}{|p{4cm}|p{12cm}|}
\hline
{\cellcolor{Color1}Задание} & Описание на проекта \\
\hline
\hline
{\cellcolor{Color1}Дисциплина} & Анализ на софтуерните изисквания 2023-2024 \\
\hline
\end{xltabular}

\begin{xltabular}{\textwidth}{|p{0.5cm}|p{5.5cm}|p{4.5cm}|p{3cm}|p{1.2cm}|}
\hline
\rowcolor{Color1}\multicolumn{5}{|l|}{Участници в проекта} \\
\hline
\rowcolor{Color2}\textnumero & Име и фамилия & Факултетен \textnumero & Специалност & Курс \\
\hline
1 & Виктор Христов & 1MI0600098 & СИ & III \\
\hline
2 & Пламена Стоянова & 3MI0600019 & СИ & III \\
\hline
3 & Божидар Горанов & 0MI0600022 & СИ & III \\
\hline
4 & Мария Иванова & 9MI0600045 & СИ & III \\
\hline
5 & Николай Костандиев & 6MI0600046 & СИ & III \\
\hline
\end{xltabular}

\begin{xltabular}{\textwidth}{|p{4cm}|p{12cm}|}
\hline
\cellcolor{Color1}Име на група & \parbox[t]{12cm}{The Pentagon} \\
\hline
\end{xltabular}

\begin{xltabular}{\textwidth}{|p{4cm}|p{12cm}|}
\hline
\cellcolor{Color1}Име на проекта & \parbox[t]{12cm}{Система за онлайн резервации на медицински прегледи (MedAPPoint)} \\
\hline
\end{xltabular}

\section{Обобщение}
На всеки му се е налагало посещение при лекар. За съжаление, на повечето места винаги трябва да се чака на опашка с часове и дори не е сигурно дали проблемът е сериозен, както и дали нашите оплаквания са в компетенциите на специалиста, когото сме посетили. Нашата система предлага решение на този проблем, като предоставя възможност на потребителите да си запазват часове онлайн, при различни медицински специалисти. В случай, че не е сигурно дали даден проблем се отнася за определен специалист, потребителите имат опцията за онлайн консултация. Обикновено се налага да носим със себе си най-различни медицински документи - изследвания, прилагани лечения, приемани медикаменти и т.н. Това вече няма да бъде необходимо, тъй като всеки потребител ще има възможността да качва и съхранява своите медицински документи на сървъра. При нужда от съдействие с работата в системата, потребителите ще имат достъп до 24-часов support чат.  Преди да направи своята резервация, всеки потребител ще има възможността да прочете отзивите и оценките за различните специалисти и медицински заведения (болници, клиники и др.) След като е посетил даден специалист/медицинско заведение, потребителят ще може да остави ревю и да оцени отношението и качеството на услугата, за да помогне при избора на следващите пациенти.\\

\section{Акаунти}
Различните видове акаунти биват:
\begin{enumerate}
\item Administrator - Управлява системата и създава акаунтите за медицинските центрове.
  \item Customer Support - Потребителите се свързват с тях посредством вграден чат, при въпроси и проблеми при работа със системата.
  \item Medical Center - Представлява медицинско заведение. Може да добавя лекари, които работят в медицинското заведение и да получава отзиви от потребители, които са го посетили. Също позволяват на пациенти да запазват прегледи на място.
  \item Doctor - Медицинско лице, което се регистрира и предлага своите услуги в системата. Може да получава отзиви и оценки от потребители, които са преминали преглед при него.
  \item Premium user - Премиум потребител, който има достъп до всички функционалности на системата, заедно с допълнителни такива, срещу заплащане.
  \item User - Обикновен потребител, който може да прави резервации за часове на място и за онлайн консултации, да оставя отзиви и оценки, както и да използва всички функционалности на системата (с изключение на премиум функциите).
  \item Guest user - Нерегестриран потребител, който има възможност да разглежда системата и отзивите, без да може да прави резервация, да се възползва от онлайн консултация, да оставя рецензии за специалисти и лечебни заведения.
\end{enumerate}

\end{document}
